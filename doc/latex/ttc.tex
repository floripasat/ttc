%*********************************************************
% Federal University of Santa Catarina (UFSC)
% 
% Author: 				Gabriel Mariano Marcelino
% 
% Created on: 			13/12/2017
% Last modification: 	15/12/2017
%*********************************************************

\documentclass[12pt]{book}
\usepackage[a4paper,left=2.5cm,right=2.5cm,top=2.5cm,bottom=2.5cm]{geometry}
\usepackage[T1]{fontenc}  
\usepackage[utf8]{inputenc} 
\usepackage[english]{babel}
\usepackage{ae}
\usepackage{graphicx}
\usepackage[hidelinks]{hyperref}
\usepackage{fancyhdr}
\usepackage{subfigure}
\usepackage{nomencl}
\usepackage{float}
\usepackage{titlesec}
\usepackage{booktabs}
\usepackage{emptypage}
\usepackage{lettrine}
\usepackage{tabularx}
\usepackage{marvosym}

\title{Telemetry, Tracking and Command Module of the FloripaSat Project}
\author{Gabriel Mariano Marcelino}
\date{13/12/2017}

% File metadata
\hypersetup
{
    pdfauthor	={Gabriel Mariano Marcelino},
    pdfsubject	={Telemetry, Tracking and Command Module Documentation},
    pdftitle	={Telemetry, Tracking and Command Module of the FloripaSat Project},
    pdfkeywords	={Cubesats, TTCs, Telecomunications}
}

% URLs font style
\urlstyle{same}

% First chapter page style
\titleformat{\chapter}[display]
	{\bfseries\Large}
	{\filright\MakeUppercase{\chaptertitlename} \Large\thechapter}
	{1ex}
	{\titlerule\vspace{1ex}\filleft}
	[\vspace{1ex}\titlerule]

% Header style
\pagestyle{fancy}
\fancyhf{}
\fancyhead[RO]{Telemetry, Tracking and Command Module}
\fancyhead[LE]{\nouppercase{\leftmark}}
\fancyfoot[RO]{\thepage}
\fancyfoot[LE]{\thepage}
\renewcommand{\footrulewidth}{0.5pt} 

% List of abbreviations
\makenomenclature
\setlength\nomlabelwidth{1.5cm}

% Bibliography style
\bibliographystyle{unsrt}

% Table cell size limit and alignment
\newcolumntype{L}[1]{>{\raggedright\arraybackslash}p{#1}}
\newcolumntype{C}[1]{>{\centering\arraybackslash}p{#1}}
\newcolumntype{R}[1]{>{\raggedleft\arraybackslash}p{#1}}

\begin{document}

\begin{titlepage}

%****************************************************
%****************************************************
%-- TITLE PAGE --------------------------------------
%****************************************************
%****************************************************
\thispagestyle{empty}

\begin{flushleft}
FLORIPASAT - TTC-DOC - REV1
\end{flushleft}

\begin{figure}[!ht]
	\begin{flushleft}
		\includegraphics[width=5cm]{figures/floripasat.png}
	\end{flushleft}
\end{figure}

\begin{flushleft}
\Huge{\textbf{Telemetry, Tracking and Command Module of the FloripaSat Project}}
\rule[0pt]{\textwidth}{5pt}
\end{flushleft}

\vspace{0.2cm}

\begin{flushleft}
\textit{Module Documentation} \\
\textit{GSE, Federal University of Santa Catarina, Florianópolis - Brazil}
\end{flushleft}

\vfill
\vfill

\begin{flushright}
December 2017
\end{flushright}

\pagenumbering{roman}
\setcounter{page}{1}

\end{titlepage}

\cleardoublepage

%****************************************************
%****************************************************
%-- AUTHOR PAGE -------------------------------------
%****************************************************
%****************************************************

\thispagestyle{empty}

\begin{center}

\textbf{FloripaSat Project, Telemetry, Tracking and Command Module Documentation}

\textit{December, 2017}

\vspace{1cm}

\textbf{Project Manager:}

Eduardo Augusto Bezerra

\vspace{1cm}

\textbf{Author:}

Gabriel Mariano Marcelino

\vspace{1cm}

\textbf{Contributing Authors:}

Anselmo Luis da Silva Junior \\
Marcelo Daniel Berejuck \\
Sara Vega Martinez \\

\vspace{1cm}

\textbf{Layout:}

Gabriel Mariano Marcelino

\end{center}

\vspace{8cm}

\begin{figure}[!h]
%\begin{wrapfigure}{l}{0.25\textwidth}
	\begin{center}
		\includegraphics[width=0.25\textwidth]{figures/by-sa.eps}
	\end{center}
\end{figure}
%\end{wrapfigure}

\textcopyright\  2017 by Federal University of Santa Catarina. Telemetry, Tracking and Command Module of the FloripaSat Project. This work is licensed under the Creative Commons Attribution-ShareAlike 4.0 International License. To view a copy of this license, visit http://creativecommons.org/licenses/by-sa/4.0/.

%****************************************************
%****************************************************
%-- ABSTRACT ----------------------------------------
%****************************************************
%****************************************************

\chapter*{Abstract}

This document...

\smallskip
\noindent \textbf{Keywords:} Cubesats. Embedded systems. Telecomunications.

%****************************************************
%****************************************************
%-- TABLE OF CONTENTS -------------------------------
%****************************************************
%****************************************************
\tableofcontents

%****************************************************
%****************************************************
%-- LIST OF FIGURES ---------------------------------
%****************************************************
%****************************************************

\listoffigures
\addcontentsline{toc}{chapter}{List of Figures}

%****************************************************
%****************************************************
%-- LIST OF TABLES ----------------------------------
%****************************************************
%****************************************************

\listoftables
\addcontentsline{toc}{chapter}{Lista of Tables}

%****************************************************
%****************************************************
%-- NOMENCLATURE ---------------------------
%****************************************************
%****************************************************

\printnomenclature
\addcontentsline{toc}{chapter}{Nomenclature}

%****************************************************
%****************************************************
%-- INTRODUCTION ------------------------------------
%****************************************************
%****************************************************

\chapter{Introduction}

\pagenumbering{arabic}

\lettrine{I}{ntroduction}...

\nomenclature{\textbf{TTC}}{Telemetry, Tracking and Command.}
\nomenclature{\textbf{PCB}}{Printed Circuit Board.}
\nomenclature{\textbf{USB}}{Universal Serial Bus.}

\cite{site}.

\section{Module Requirements}

%****************************************************
%****************************************************
%-- HARDWARE ----------------------------------------
%****************************************************
%****************************************************

\chapter{Hardware}

\lettrine{T}{he} TTC board is composed by the following main components:

\begin{itemize}
	\item MSP430F6659, as the beacon microcontroller.
	\item RF4463F30, as the radio module for the beacon and the telemetry link.
\end{itemize}

In the figure \ref{fig:ttc-board}, ...

\begin{figure}[!h]
	\begin{center}
		\includegraphics[width=0.75\textwidth]{figures/ttc_board.png}
		\caption{TTC PCB.}
		\label{fig:ttc-board}
	\end{center}
\end{figure}

\section{General Diagram}

In the figure \ref{fig:hardware-diagram}, a general hardware diagram can be seen.

\begin{figure}[!h]
	\begin{center}
		\includegraphics[width=\textwidth]{figures/hardware_diagram.pdf}
		\caption{Hardware diagram of the TTC module.}
		\label{fig:hardware-diagram}
	\end{center}
\end{figure}

\section{Main Components}

M...

\subsection{Microcontroller}

The beacon microcontroller is the MSP430F6659IPZR \cite{msp430f6659}. Its main characteristics can be found in the table \ref{tab:msp430f6659-info}.

\nomenclature{\textbf{CPU}}{Central Processing Unit.}
\nomenclature{\textbf{RAM}}{Random Access Memory.}
\nomenclature{\textbf{GPIO}}{General Purpose Input/Output.}
\nomenclature{\textbf{I$^{2}$C}}{Inter-Integrated Circuit.}
\nomenclature{\textbf{SPI}}{Serial Peripheral Interface.}
\nomenclature{\textbf{UART}}{Universal Asynchronous Receiver/Transmitter.}
\nomenclature{\textbf{DMA}}{Direct Memory Access.}
\nomenclature{\textbf{ADC}}{Analog-To-Digital Converter.}
\nomenclature{\textbf{BSL}}{Bootstrap Loader.}

\begin{table}[!h]
	\begin{center}
		\begin{tabular}{lc}
			\toprule[1.5pt]
			\textit{Characteristic} & \textit{Value} \\
			\midrule
			CPU & MSP430 \\
			Frequency & Up to 20 MHz \\
			Non-volatile memory & 512 kB \\
			RAM & 66 kB \\
			GPIO pins & 74 \\
			I$^{2}$C & 3 \\
			SPI & 6 \\
			UART & 3 \\
			DMA & 6 \\
			ADC & ADC12-12ch \\
			Comparators & 12 inputs \\
			Timers - 16-bit & 4 \\
			Multiplier & $32 \times 32$ \\
			BSL & USB \\
			Min $V_{cc}$ & 1,8 V \\
			Max $V_{cc}$ & 3,6 V \\
			Active Power & $360\ \mu A/MHz$ \\
			Standby Power (LMP3) & $2,6\ \mu A$ \\
			Wakeup Time & $3\ \mu s$ \\
			Operating Temperature Range & -40 to 80 $^{\circ}C$ \\
			\bottomrule[1.5pt]
		\end{tabular}
		\caption{MSP430F6659 features.}
		\label{tab:msp430f6659-info}
	\end{center}
\end{table}

\subsection{Radio Modules}

The NiceRF RF4463F30 \cite{rf4463f30} is a transceiver module based on the Silicon Labs Si4463 \cite{si4463} radio. This module also contains a PA module to increase the output power up to 31 dBm.

\subsubsection{Si4463}

\begin{table}[!h]
	\begin{center}
		\begin{tabular}{lcc}
			\toprule[1.5pt]
			\textit{Characteristic} & \textit{Value} & \textit{Unit} \\
			\midrule
			Frequency range & 119-1050 & MHz \\
			Receiver sensitivity & -126 & dBm \\
			Modulation & (G)FSK, 4(G)FSK, (G)MSK and OOK & - \\
			Max. output power & +20 & dBm \\
			PA support & +27 to 30 & dBm \\
			Ultra low current powerdown modes & 30 (shutdown), 50 (standby) & nA \\
			Data rate & 100 bps to 1 Mbps & - \\
			Power supply & 1,8 to 3,6 & V \\
			TX and RX FIFOs & 64 bytes for each or 129 bytes shared & - \\
			\bottomrule[1.5pt]
		\end{tabular}
		\caption{Si4463 features.}
		\label{tab:si4463-info}
	\end{center}
\end{table}

\section{External Connections}

This section describes the external available connections of the TTC module.

In the figure \ref{fig:connections-ref}, all the external connections are enumerated.

\begin{figure}[!h]
	\begin{center}
		\includegraphics[width=0.75\textwidth]{figures/ttc_board_pins}
		\caption{External connections on the board.}
		\label{fig:connections-ref}
	\end{center}
\end{figure}

A brief description of each connection is presented in the table \ref{tab:connections-ref}.

\begin{table}[!h]
	\begin{center}
		\begin{tabular}{L{1.3cm} C{3cm} C{10cm}}
			\toprule[1.5pt]
			\textit{Number} & \textit{Connector} & \textit{Description} \\
			\midrule
			1 & Male pin header ($1 \times 2$) & UART TX \MVAt 4800 bps. These pins transmit the beacon packets over a serial connection (It is enable in the configuration file, setting the BEACON\_RADIO variable as UART\_SIM). \\
			2 & Male pin header ($1 \times 2$) & Debug UART TX/RX \MVAt 115200 bps. These pins transmit a description of the main events of the beacon software during it's execution. This feature is only available in DEBUG\_MODE. \\
			3 & Male PicoBlade$^{TM}$ ($\times 6$) & JTAG and Debug. This connection contains the relevant pins of the connectors 2 and 4. \\
			4 & Male pin header ($2 \times 2$) & MSP430 JTAG. This connection is for programming the uC code, using a MSP-FET debugger. \\
			5 & Male PicoBlade$^{TM}$ ($\times 6$) & Antenna I2C. I2C bus for a communication channel with the antenna module. \\
			6 & Male pin header ($1 \times 2$) & Power supply jumper. With a jumper, the beacon microcontroller power source comes from the JTAG connector. Without a jumper, the uC power supply comes from a pin of the PC104 connector. \\
			7 & Female Angled MCX & 437 MHz band RF signal (Goes to the antenna module). \\
			8 & Female Angled MCX & 145 MHz band RF signal (Goes to the antenna module). \\
			9 & Male/Female PCI-104 & PCI-104. Power supply and communication buses with others stacked up modules. \\
			\bottomrule[1.5pt]
		\end{tabular}
		\caption{External connections description.}
		\label{tab:connections-ref}
	\end{center}
\end{table}

The connections 1, 2, 4 and 6 were designed to be used during the software development stage, and not during the satellite operation.

\subsection{PCI-104 Pins}

The table \ref{tab:pci104-ref} describes the PCI-104 connector used pins. The first column is the row number of the connector, and the remaining columns are the respective columns (Named as H1A, H1B, H2A and H2B respectively). If the pin has no description, it is not connected to the TTC board.

\begin{table}[!h]
	\begin{center}
		\begin{tabular}{L{1cm} C{3cm} C{3cm} C{3cm} C{3cm}}
			\toprule[1.5pt]
			\textit{Row} & \textit{H1A} & \textit{H1B} & \textit{H2A} & \textit{H2B} \\
			\midrule
			1 & GND & GND & GND & GND \\
			2 & GND & GND & GND & GND \\
			3 & - & - & UART RX \MVAt 4800 bps from the EPS module. & - \\
			4 & Telemetry radio GPIO0 & Telemetry radio GPIO1 & - & - \\
			5 & Telemetry radio GPIO2 & Enable beacon radio power supply & - & - \\
			6 & Telemetry radio SDN & - & OBDH communication (SPI MOSI) & OBDH communication (SPI clock) \\
			7 & - & - & OBDH communication (SPI chip select) & OBDH communication (SPI MISO) \\
			8 & - & - & - & - \\
			9 & - & - & - & - \\
			10 & - & - & - & - \\
			11 & - & - & - & - \\
			12 & - & - & - & - \\
			13 & - & - & - & - \\
			14 & - & - & Beacon uC power supply (3,3 V/50 mA) & 3,3 V beacon uC power supply (3,3 V/50 mA) \\
			15 & GND & GND & GND & GND \\
			16 & GND & GND & GND & GND \\
			17 & - & - & - & - \\
			18 & Telemetry radio SPI clock & - & - & - \\
			19 & Telemetry radio SPI MISO & - & - & - \\
			20 & Telemetry radio SPI MOSI & Telemetry radio SPI chip select & - & - \\
			21 & - & - & - & - \\
			22 & - & - & - & - \\
			23 & - & - & - & - \\
			24 & - & - & - & - \\
			25 & Telemetry radio power supply (5 V/500 mA) & - & - & - \\
			26 & Beacon radio power supply (5 V/500 mA) & - & - & - \\
			\bottomrule[1.5pt]
		\end{tabular}
		\caption{PCI-104 connector reference.}
		\label{tab:pci104-ref}
	\end{center}
\end{table}

%****************************************************
%****************************************************
%-- SOFTWARE ----------------------------------------
%****************************************************
%****************************************************

\chapter{Software}

\lettrine{S}{oftware}...

\nomenclature{\textbf{HAL}}{Hardware Abstraction Layer.}

\begin{figure}[!h]
	\begin{center}
		\includegraphics[width=0.5\textwidth]{figures/beacon_software_layers.pdf}
		\caption{Beacon software stack-up.}
		\label{fig:beacon-software-layers}
	\end{center}
\end{figure}

\section{Flowcharts}

\begin{figure}[!h]
	\begin{center}
		\includegraphics[width=0.25\textwidth]{figures/beacon_main_flowchart.pdf}
		\caption{Main flowchart of the beacon software.}
		\label{fig:beacon-main-flowchart}
	\end{center}
\end{figure}

\nomenclature{\textbf{ISR}}{Interruption Service Routine.}

\begin{figure}[!h]
	\begin{center}
		\subfigure[OBDH communication ISR flowchart.\label{fig:beacon-obdh-isr-flowchart}]{\includegraphics[width=0.4\textwidth]{figures/beacon_obdh_isr_flowchart.pdf}}
		\qquad
		\subfigure[EPS communication ISR flowchart.\label{fig:beacon-eps-isr-flowchart}]{\includegraphics[width=0.4\textwidth]{figures/beacon_eps_isr_flowchart.pdf}}
		\caption{OBDH and EPS modules comunication ISRs routines.}
		\label{fig:obdh-eps-isr-flowchart}
	\end{center}
\end{figure}

%****************************************************
%****************************************************
%-- TESTS -------------------------------------------
%****************************************************
%****************************************************

\chapter{Tests}

\lettrine{T}{his}...

\section{RF Signal Power}

P...

%****************************************************
%****************************************************
%-- CONCLUSION --------------------------------------
%****************************************************
%****************************************************

\chapter{Conclusion} \label{ch:conclusion}

\lettrine{C}{onclusion}...

%****************************************************
%****************************************************
%-- REFERENCES --------------------------------------
%****************************************************
%****************************************************

\bibliography{references/site}
\addcontentsline{toc}{chapter}{Bibliography}

\end{document}
